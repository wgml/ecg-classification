\section*{Podsumowanie}

W ramach projektu zaproponowano klasyfikatory sygnału $EKG$ opierające się na metodach $KNN$ i $ENN$. Zbudowano prototypy algorytmów w języku $Matlab$ a następnie zaprojektowano kompatybilną z prototypami aplikację w języku $C++$. Przeniesienie części algorytmicznej było możliwe dzięki wykorzystaniu bibliotek wspierających obliczenia na wektorach i macierzach. 
Przeprowadzono szereg testów mających na celu zbadanie skuteczności działania algorytmów oraz wyznaczono wskaźniki czułości i swoistości klasyfikatorów. Na podstawie uzyskanych wyników przyjęto większą użyteczność klasyfikatora $KNN$ w omawianym zadaniu. Średnia skuteczność jest wyższa o kilka punktów procentowych względem $ENN$ przy kilkukrotnie niższej złożoności czasowej aplikacji. Algorytm $ENN$ pozwolił jednak na zwiększenie skuteczności klasyfikacji w przypadku klas o niewielkiej liczności w zbiorze uczącym. \textbf{jeszcze jakieś zalety?}
Badane metody klasyfikacji mogą pracować równolegle na jednym zbiorze uczącym, co pozwala na wykorzystanie technik programowania na systemy wieloprocesorowe. Wykorzystany w ramach projektu interfejs $OpenMP$ pozwolił na kilkukrotne zmniejszenie czasu potrzebnego na analizę zbioru testowego względem aplikacji jednowątkowej. Dalszą poprawę wydajności mogłoby dać wykorzystanie biblioteki $OpenCL$ lub pokrewnej.
